
\documentclass[10pt,epsf]{article}

\usepackage{float} \newcommand{\vm}{\vspace{0.2cm}} \newcommand{\vl}{\vspace{0.4cm}} \newcommand{\per}{\hspace{.2cm}}

\input{/Users/matpitka/tgdmaterials/bibdats/auxiliaries} \setlength{\intextsep}{-1ex}

% remove extra space above and below in-line float

%\input{/Users/matpitka/tgdmaterials/bibgrafiat/bibdefs}

\title{{\bf About the structure of Dirac propagator in TGD}} \author{}

\begin{document}

\maketitle

\input{/Users/matpitka/tgdmaterials/addresses/address1}


\begin{abstract}

In this article the notion of fermion propagator  is discussed. It is found that the construction is much more than a mere computational challenge. There are two alternative approaches. Fermionic propagation could correspond to a) a 4-D  or lower-dimensional propagation at the space-time level for the induced spinor fields as analog of massless propagation or b) to 8-D propagation in $H$ between points belonging to the space-time surface. 

For the option b) the separate conservations of baryon and lepton number requires fixed $H$-chirality so that the spinor mode is sum of products of $M^4$ and $CP_2$ spinors with fixed $M^4$ and $CP_2$ chiralities whose product is +1 or -1.  This implies that $M^4$ propagation is massless. The Euclidean signature of $CP_2$ in turn suggests that there is no propagation in $CP_2$ and the $CP_2$ projections  $s_1$ and $s_2$ for the ends of the propagator line are the same. This allows a very simple form for the matrix elements of the propagator and the condition  $s_1=s_2$ implies that the propagation from $s_1$ is possible to    a discrete set of points $s_2$. This has direct relevance for the understanding of color confinement and more or less implies the intuitively deduced TGD based model for elementary particles.

Although the option a) need not provide a realistic propagator, it could provide a very useful semiclassical picture. If the condition $s_1=s_2$ is assumed, fermionic propagation along light-like geodesics of $H$ is favored and in accordance with the model for elementary particles. This allows a classical  space-time picture of particle massivation by p-adic thermodynamics and    color confinement.


Also the interpretational and technical  problems related to the construction of 4-D variants of super-conformal representations having spinor modes as ground states and to the p-adic thermodynamics are discussed.

\end{abstract}

\tableofcontents


\section{Introduction}

In this article the  identification and calculation of the  fermion propagator $S_F(h_1,h_2)$ in $H=M^4\times CP_2$ is discussed in detail. Also the definition
of fermion propagator at space-time surface is considered.

\subsection{Basic questions}

Let us summarize the basic physical picture first.  The following picture is only one of the many variants  and must of course be taken with reservations.

\begin{enumerate}

\item The TGD view of  fundamental interactions  \cite{btart}{TGD2024I,TGD2024II} differs from that of the standard model.  The notion of color differs from that of QCD and electroweak gauge potentials correspond to the components of the induced spinor connection of $CP_2$ \cite{allb}{tgdnewphys1,tgdnnewphys2}.  A strong  correlation between mass, color quantum numbers and electroweak quantum numbers at  the fundamental level corresponding to space-time surfaces is predicted \cite{btart}{isospinbreak}. The standard view emerges as an approximation at the QFT limit.

\item In $CP_2$ degrees of freedom, well-defined $H$ chirality implies that the  $H$ propagator $D_F(m_1,s_1),(m_2,s_2))$   is reduced to  a bilocal inner product for the two modes with the same  $CP_2$ chirality defined by the propagator factor associated with the propagator. An important constraint is that the end  points are restricted to the space-time surface. If the propagator reduces to the delta function in $CP_2$ degrees of freedom, as the Euclidean metric suggests, one has $s_1=s_2$ with very powerful consequences. Second important point is that in $M^4$  the propagation is between identical $M^4$ chiralities by the separate conservation of lepton and quark quantum numbers.

\item What  does one mean  with a virtual fermion now?  Propagation in $H$ is the natural  gemetric definition for virtuality. The first naive guess was that the virtual fermions  correspond to masses $p^2=\Lambda_n^2$, $\Lambda_n^2$ is the eigenvalue of the $CP_2$ Dirac operator,   as predicted by the $H$ Dirac equation in $H$.  

This however leads to a conclusion that the propagator does not have these masses as poles unless the normalization of the mass shell states is such that poles are there. Furthermore, these masses do not correspond to the masses of physical fermion, which are generated by p-adic thermodynamics. This picture looks therefore unphysical.

\item The condition  that quark and lepton numbers are separately conserved forces us to assume a fixed $H$-chirality so that fermions are massless in 8-D sense. This implies that the spinor modes are superpositions of modes for which $M^4$ chirality and $CP_2$ chirality are well-defined and either the same or opposite depending on whether quark or lepton is in question. Fixed $M^4$ chirality requires a massless propagation in $M^4$. This  is a rather dramatic prediction and means that propagation completely separates from the properties and construction of states.

$CP_2$ has Euclidean signature of metric, which suggests that propagation is not possible at all. The propagator would reduce  to a delta function in $CP_2$: the end points $s_1$ and $s_2$ would be identical for $CP_2$ propagation. A second dramatic prediction is that,  the condition $s_1=s_2$  allows for a fixed point $s_1$ only a discrete set of final points.

\item  The  propagation in $H$ is analog of off-mass shell propagation  as off-space-time surface propagation. The notion of induced spinor structure however suggests that it  might make  sense to speak of propagation along the space-time surface as analog of on-mass-shell propagation.  

The induced Dirac equation in $X^4$ is analogous to the massless Dirac equation.  The condition $s_1=s_2$ however suggests that this propagation is along light-like geodesics   of $H$, whose projection is a geodesic circle of $CP_2$. This would give a classical geometric correlate for the  fermion propagation even if the space-time propagator might not be needed.

\end{enumerate}


\subsubsection{How do the physical color quantum numbers emerge?}

 It must be taken into account that physical fermions are either color singlets or triplets, not arbitrarily high color partial waves satisfying only the triality condition $t=1$ for quarks and $t=0$ for leptons.

This is achieved in  two ways. 

\begin{enumerate}

\item Neutrino-antineutrino $\nu_L\overline{\nu}_R$ pair screens the electroweak charges above weak boson Compton length and could also screen the anomalous parts of color charges.

\item  The solutions   Dirac equation are ground states of   conformal representations in which   the  action of a conformal  scaling generator generates higher excitations,  which can have color quantum numbers which  add to those of color partial waves to give rise to color singlets and triplets.

\end{enumerate}

p-Adic thermodynamics  \cite{btart}{padmass2022} describes the thermodynamics of the scaling generator, and gives excellent predictions for the  particle masses so that conformal algebra is involved in any case. 

For the proposed form of the $H$ propagator, the construction of the quantum states separates completely from that for the propagator.


\subsubsection{How  could the confinement scale emerge?}

In QCD, confinement    occurring on some $M^4$   scale, the confinement scale, is a fundamental concept. Free  quarks are the basic dynamical entities  below the  confinement scale and  transform to hadrons above this scale. 

\begin{enumerate}

\item   In attempts to  understand this,   the space-time representation  of $H$ propagator $D_F(m_1,s_1),(m_2,s_2))$ for the massa propagator $M^4$ degrees of freedom should be used (see \href{https://en.wikipedia.org/wiki/Propagator}). 



One can consider two options depending on whether the propagation occurs in $H$ or  in $X^4$. What is expected to happen in the case of $H$ propagation, is the following.

\begin{enumerate}

\item At least in  short scales when $a^2=(m_1-m_2)^2$ ($a$ is Minkowski distance is small enough,  the propagator is essentially a delta function $\delta (s_1,s_2)$ in $CP_2$ degrees of freedom. This would be a phase where the quarks are free and effectively massless. What is remarkable is that the  $s_1=s_2$ condition implies that for a given $s_1$ there is only a discrete set of points that the propagator can connect! At partonic orbits for which $CP_2$ projection is fixed the number of these points would be large.
 

\item  If the condition $s_1=s_2$  holds true quite generally,  the classical non-determinism of the space-time dynamics implies that the number of points satisfying these conditions is  reduced  during  the time evolution and   makes propagation impossible for large values of $a^2$. This would mean  quark confinement. This of course also applies to possible colored excitations of leptons predicted by TGD and for which there is some evidence \cite{allb}{leptc}.
\end{enumerate}

\item It is reasonable to expect that the p-adic length scale determines the scale at which the confinement occurs. This is where the algebraic geometry of the spacetime surface comes into play.

\end{enumerate}

Here  holography= holomorphy vision  suggests a concrete approach.

\begin{enumerate}

\item  In the holography= holomorphy vision \cite{btart}{Frenkel,Langlands2025},  space-time surfaces corresponds to the roots of  polynomial pair $f=(f_1,f_2): H\rightarrow C^2$ of generalized complex coordinates of $H$  (one hypercomplex coordinate and 3 complex coordinates). They allow as dynamic symmetries the maps $g=(g_1,g_2): C^2\rightarrow C^2$ defined as functional compositions $f\rightarrow g\circ f$. 

One can define prime polynomial pairs  $f$ as those which do not allow a decomposition $f=g\circ h$. The prime polynomial pairs $g=(g_1,g_2$ can be defined similarly. The pairs $(g_1,Id)$ are of special interest and in this case the prime polynomials $g_1(p)$ have prime degree $p$. 

\item One should study the $H$  propagator $D_F(m_1,s_1),(m_2,s_2))$ for the polynomials $g_p$ and the functional iterates $g_p^{\circ k}\circ f$. The integration over $CP_2$ projections of  the end points gives a bilocal inner product of the modes. The constraint $s_1=s_2$ reduces it to an  ordinary inner product for the $CP_2$ spinor modes.

The number of point pairs  satisfying the constraint $s_1=s_2$ depends on the space-time surface.   The functional iterates  of $g_p$ involve   classical non-determinism identifiable as p-adic non-determinism \cite{btart}{HHtwistor} so that the number space-time points satisfying the constraint $s_1=s_2$ decreases as the number $k$ of iteration steps increases. This would make propagation  over too long scales impossible  p-Adic length scale could correspond to this scale. Note that very large primes $p$  are involved: for the electron  one has $p=M_{127}=2^{127}-1$.

The confinement scale  can be identified as a p-adic length scale for quarks or hadrons, which is determined by the  thermal mass  predicted by p-adic thermodynamics. The larger  the value of $p$,   the longer the confinement scale. 

\end{enumerate}


Concerning the option based on $X^4$ propagation two observations can be made.


\begin{enumerate}

\item The possibly existing  $X^4$ propagator would be the analog of a massless propagator  since  the Dirac equation for induced spinor fields in $X^4$ is a generalization of the massless Dirac equation. This notion makes sense also for the lower dimensional surfaces of $X^4$ and 1-D light-like curves are  natural candidates for fermion lines.

\item An observation of possible relevance is that  the masses of quarks are effectively zero for the  propagation along the light-like $M^4$ geodesics along parton surfaces. 

\end{enumerate}




\subsubsection{How the effective mass scale of a fermion is determined?}

 There are two basic questions to be answered. Is the p-adic mass scale determined by the spectrum of $CP_2$ masses or by p-adic thermodynamics? Does the spectrum  of $CP_2$ masses or p-adic thermal masses appear at the level of propagator or only at the level of initial and final states of propagation?
 
\begin{enumerate}

\item A naive assumption would be that the quark and lepton masses appear in the propagator depend on the spectrum of color excitations $m^2=\lambda_n^2$ for the fermions involved ($\nu_L\overline{\nu}_R$ for the second option). Several arguments suggest this cannot be the case.

p-Adic thermodynamics, in which the  basic parameters characterizing the particle are the p-adic prime  $p$  and the p-adic temperature $T_p= 1/n$,  works excellently so that it seems that $CP_2$ mass spectrum does not make itself visible in the propagator. In p-adic thermodynamics, a large $p \simeq  2^k$,   simplifies enormously and something similar can be expected now as well.

\item    If the propagation takes in $H$, it is massless for a fixed $H$ chirality so that  the eigenvalue spectrum  of $D(CP_2)$  is not visible in propagation.  This separation between the structure of states and propagator  would be a huge simplification.

\item  Concerning $X^4$ propagation, the semiclassical description as light-like $H$ geodesics of the geodesic manifold $M^4\times S^1\subset M^4\times CP_2$ could be considered as a model. The masslessness of $H$ would no longer imply   masslessness in  $M^4$. One could say that p-adic thermal excitations  deform the light-like geodesics of $M^4$ to those of $M^4\times S^1\subset H$.  Also this suggests that the masses are not visible in the $X^4$  propagator as also the fact that they do not appear in the $X^4$ Dirac operator suggests. 

\end{enumerate}



\subsection{Plan of the article}

The plan of the article is as follows.

\begin{enumerate}

\item I have considered several approaches to the calculation of the fermion propagator and    in the following I describe  the most elegant looking view about the fermion  propagation as propagation in $H$.  Assuming a fixed $H$ chirality, the propagator reduces to massless propagator in $H$ and is proportional to  delta function in $CP_2$ so that no propagation occurs in $CP_2$ degrees of freedom. This has powerful consequences and  the existing intuitive picture about fundamental fermion lines as light-like geodesics at partoic 2-surfaces follows. One can also understand color confinement in this picture.

\item The 4-D propagation along $X^4$ involves induced spinor structure and  my  intuitive feeling is that  also this perspective is   important, at least as semiclassical model. The condition $s_1=s_2$ for $H$ propagation  suggests that fermion propagation is always 1-dimensional and along a light-like geodesic of $H$. This would give a very concrete classical view of the propagation even if a 4-D propagator were not useful. In particular, the p-adic thermal massivation could be modelled  semi-classically.

\item Also the problems related to the  origin of p-adicity and interpretational  problems of p-adic thermodynamics and to  the generalized conformal invariance will be discussed in the Appendix. 

\end{enumerate}






\section{How does the propagation take place?}


I have considered two alternative views concerning  what  fermion propagation could mean. 

\begin{enumerate} 

\item Fermions propagate along the 4-D space-time surface  $X^4$ as induced spinor fields and the propagator is defined by the analog of 4-D  massless Dirac operator.  

\item Fermions propagate in $H$ and virtuality in geometric sense could mean that they can  leave the space-time surface during the propagation. Off mass-shell property would mean off-space-time surface.

\end{enumerate}

Could the two interpretations for the  fermion propagation be consistent? The  space-time propagation as a 4-D propagation is geometrically much more restricted.  Could the propagator code for the information about initial and final quantum states  as a kind of quantum-classical correspondence?



\subsection{Leptons and quarks as quantized free spinor fields in $H$}

It is good to start with the description of $H$ spinor fields and of the $H$ Dirac equation. This topic is also discussed in Appendix A.

\subsubsection{Leptons and quarks as $H$ spinors with different $H$ chiralities}

Consider first the general theory.


\begin{enumerate}

\item The second quantized spinor field in $H=M^4\times CP_2$    can be written as superpositions of modes multiplied with a creation or annihilation operator.

\item Suppose that the spinor fields of $H$ have a definite H-chirality so that they are either leptonic or quark-like, not a superposition of the two: this would violate the superselection rule of charge and fermion number.

One has

\begin{eqnarray}
\Gamma_9\Psi_{\epsilon} = \epsilon \Psi_{\epsilon} .
\end{eqnarray}

\noindent $\epsilon =+1/-1$ is the H-chirality and quarks and leptons have opposite chiralities.

\item The Dirac equation is satisfied by the spinor modes of $H$. They  are massless in $H$ but not in $M^4$(!). This is essential for the  twistorization and the divergence-free nature of the theory.

The Dirac operator is

\begin{eqnarray}
D= D(M^4)+ D(CP_2), 
\end{eqnarray}

\noindent   where 

\begin{eqnarray}
D(M^4))=\gamma^kp_k 
\end{eqnarray}


\noindent is the Dirac operator in $M^4$ and 

\begin{eqnarray}
D(CP_2)=\Gamma^kD_k .
\end{eqnarray}


\noindent is the Dirac operator in $CP_2$.   Here $D_k$ is covariant derivative, which is defined by the spinor connection of $CP_2$ that encodes electroweak interactions. The K\"ahler gauge potential $A_k$ for leptons is $3A_k$ and for quarks $A_k$. This explains the different electric charges of quarks and leptons.

Twistorialization suggests that   $M^4$ allows the analog of K\"ahler structure whereas vielbein connection should be trivial by the flatness of the metric. K\"ahler form should satisfy the analog  self-duality meaning that locally it is analogous to  mutually orthogonal electric and magnetic fields of  the same magnitude. 


\item Quarks correspond to triality $t=1$ partial waves  and leptons to triality $t=0$ partial waves. These modes  are not simply color triplets for physical quarks and color singlets for physical leptons. For physical fermions, a mechanism  producing color triplets for quarks and singlets for leptons is needed. 

According to the first section of the Appendix,   the  color partial waves with different charges correspond to color partial waves characterized by an integer pair $(p,q)$. $p$ and $q$ can thought of as giving  the numbers of triplets and anti-triplets  in the tensor containing the representation in question. The dimension  of the color representation is  $d(p,q)= (1/2)(p+1)(1+1)(p+q+2)$.

 In the case of leptons, one has representations $(p,p)$ for neutrinos and $(3+p,p)$ for charged leptons. The dimensions are $(p+1)^2(p+2)$ neutrinos and $(3+p+1)(3+p)(p+1)$ for charged leptons. The color representations are the same for both $CP_2$ chiralities which correspond to opposite $M^4$ chiralities for a fixed $H$ chirality.

In the case of quarks one has representations $(p+1,p)$ and $(p+4,p)$ for $D$  {\it resp.}  $U$ type quarks  and the  corresponding dimensions are $1/2(p+1)(p+1)(2p+3)$ {\it resp.} $(p+6)(p+1)(p+3)$   The two $CP_2$ chiralities are transformed to each other by   $D(CP_2)$: covariantly right-handed neutrino is an  exception since it is annihilated by $D(CP_2)$. 


\item The charge assignable to  color partial wave is not a local notion characterized by a spinor index since $D(CP_2)$ mixes different isospin states,  whose identification depends on the choice of the vierbein in $CP_2$. This is understandable since the choice of the $CP_2$ spinor basis is determined apart from a local   $U(2)$ rotation. The same problem is encountered in standard model and solved by going to unitary gauge. One can ask whether it might be possible to define the counterpart of the unitary gauge as a global choice of $CP_2$ vierbein for which the $CP_2$ spinor  is constant  or covariantly constant for a given  color partial wave. 

\end{enumerate}

\subsubsection{Dirac equation in $H$}

\begin{enumerate}

\item Dirac equation in $H$ says the following

\begin{eqnarray}
D\Psi_{\epsilon}=0 .
\end{eqnarray}


\noindent Here $\epsilon =+1/-1$ refers to quark/lepton.

\item Solutions to Dirac equation can be constructed by taking a spinor $\Phi_{\epsilon}$, which is not a solution of Dirac equation, but satisfies the following conditions:

\begin{enumerate}

\item H-chirality is well defined.

\begin{eqnarray}
\Gamma_9\Phi_{\epsilon}= \epsilon \Phi_{\epsilon} .
\end{eqnarray}

\noindent H-chirality $\epsilon$ is the product  $\epsilon=\epsilon_1\times \epsilon_2$ of $M^4$ chirality $\epsilon_1$ and $CP_2$ chirality $\epsilon_2$ . It is -1 if the $M^4$ and $CP_2$ chiralities are opposite and +1 if they are the same.

Consider first the possibility that $\Phi_{\epsilon}$ is the tensor product of the $M^4$ spinor and the $CP_2$ spinor with the same or opposite chirality corresponding to quarks and leptons.

\item $\Phi_{\epsilon}$ is $M^4$  plane wave so that   $D(M^4) =\gamma^kp_k$ is effectively true.

\item $\Phi_{\epsilon}$ satisfies the square of the Dirac equation but not the Dirac equation:

\begin{eqnarray}
\begin{array}{l} D^2\Phi_{\epsilon}=0 ,\\ D(M^4)^2\Phi_{\epsilon}=p^2\Phi_{\epsilon} ,\\ D^2(CP_2)\Phi_{\epsilon}= -\Lambda^2 \Phi_{\epsilon} .\\ \end{array} \end{eqnarray}


\noindent An  eigenmode of $D(M^4)^2$ and $D^2(CP_2)$ is therefore in question. We obtain a spectrum $\Lambda_n^2$, the details of which are not relevant in this context.

The mass is quantized:

\begin{eqnarray}
p^2= \Lambda_n^2 .
\end{eqnarray}


\end{enumerate}

\item How to obtain the solution of Dirac's equation $\Psi_{\epsilon}$? It is obtained by operating on the spinor $\Phi_{\epsilon}$ with the Dirac operator $D=D(M^2)+D(CP_2)$:

\begin{eqnarray}
\Psi_{\epsilon}= D\Phi_{\epsilon} .
\end{eqnarray}


The Dirac equation holds

\begin{eqnarray}
D\Psi= D^2\Psi_0 = 0 .
\end{eqnarray}


\end{enumerate}

 A couple of comments regarding the chiralities are in order.

\begin{enumerate}

\item The spinor $\Phi_{\epsilon}$ has well-defined $M^4$ and $CP_2$ chiralities, but for the spinor $D\Phi_{\epsilon}$ the chiralities are not well-defined.

\item Although $\Phi_{\epsilon}$ can be chosen as the product of the $\Psi=\psi\otimes \phi$ definite $M^4$ of the spinor with chirality $\psi$ and the spinor with chirality $\phi$ of the definition $CP_2$, then $D\Psi$ is a superposition

\begin{eqnarray}
D(M^4) \psi\otimes \phi + \psi\otimes D(CP_2)\phi ,
\end{eqnarray}

\noindent where the H-chiralities have changed because $D(M^4)$ ($D(CP_2)$) changes the chirality of $M^4$ ($CP_2$). It can be said that both $M^4$ and $CP_2$ chiralities are mixed when the mass is non-zero.

\item It turns out that only a right-handed neutrino can have a massless state.

\end{enumerate}




\subsection{Does the propagation take place along space-time surface $X^4$?} 

For the modes of the induced spinor field  the propagation would  take  place  along  the spacetime surface and  possibly also along  a related lower-dimensional surface such as a partonic 2-surface and light-like curves along it. 

\begin{enumerate}

\item    This 4-dimensional (or  possibly lower-dimensional) propagator corresponds to the inverse  of the Dirac operator for the induced gamma matrices  as projection of  the  8-D vector formed by $H$ gamma matrices. Induced gamma matrices anticommute to the induced metric. 

\item One can also consider   modified  gamma matrices determined as contractions of the canonical momentum currents of the classical action with the gamma matrices of $H$ \cite{allb}{cspin,wcwnew}. These  do not anticommute to the induced metric but for  them  4-D  super-conformal symmetry would be exact and classical field equations would appear as a consistency condition guaranteeing hermiticity. 

For the induced gamma matrices super-conformal symmetry is violated at the vertices \cite{btart}{whatgravitons}, where the full action makes itself visible: outside the singularities minimal surface property consistent with volume action, giving rise to induced gamma matrices, holds true. 

\item The beauty  of this view (also that based on $H$ propagator) is that the constructions of the propagator and  of  the physical states  are  completely separated from each other.    The propagator is the inverse of the Dirac operator for induced spinors. For $H$ propagator the construction is easy but the situation need not be  the same now.

\item One might hope that holography= holomorphy principle \cite{btart}{Frenkel,Langlands2025,HHtwistor}    allows the explicit construction  of the propagator using complex coordinates assignable to the Hamilton-Jacobi structure  involving hypercomplex  coordinates $(u,v)$   and   complex coordinates $(w,\overline{w})$ \cite{btart}{HJ}. Hypercomplex coordinates $(u,v)$   would be the analogs of light-like coordinates and the Minkowskian signature would  make  the propagation possible.  

In the Euclidean situation the propagator would be bilinear in holomorphic and antiholomorphic  modes  $h_1=h_2$ would be a delta function singularity. Now the situation is more complex but one might hope that Wick rotation for the second complex coordinate,  transforming it to  hypercomplex coordinate,  could give the propagator.

For light-like curves at partonic orbits identified as fermionic lines, the propagator can be constructed explicitly by a semiclassical argument. 

\end{enumerate}


One can criticize this approach. 

\begin{enumerate}

\item In the case of Minkowski space it is clear what virtuality of the fermion means for the massless case: the explicit expression for  the massless  fermion propagator implies the massive virtual modes.   For virtual modes with $p^2\neq 0$   $p^k\gamma_k\Psi$ is nonvanishing.

\item What could  the  non-vanishing virtual momentum  squared correspond to in the recent case? The condition $D\Psi= (\Gamma^kD_k +\Gamma^{\overline{k}}D_{\overline{k}}\Psi=0$,  reduces for the   holomorphic modes, satisfying $D_{\overline{k}}\Psi=0$,  to the condition   $\Gamma^k\Psi=0$.  An analogous condition is true for the antiholomorphic modes.  

$\Gamma^k\Psi\neq 0$ should be true  for the virtual modes. This condition is consistent with the condition that $H$   chirality is fixed but means mixing of the $M^4$ chiralities which is the signature of massivation due to the virtual mass.

\end{enumerate}




\subsection{Does the propagation  occur in $H$?} 


The second option is that the propagation takes place in  $H$  for the spinor modes of $H$  in such a way that only  the endpoints of the propagator $D_F(h_1,h_2)$ are restricted to the space-time surface $X^4$. One could say that virtuality in geometric sense for the fermions means that they can leave the space-time surface. The propagation along space-time surface already discussed would mean the analog of on-mass-shell propagation. 



\subsubsection{Massless Dirac propagator in $M^4$}

The $M^4$ case  provides a kind of role model.    For momentum basis, the inverse of the Dirac operator  can be solved algebraically and space-time representation involves integration over virtual momenta with $p^2\neq 0$.

The momentum space - and space-time representations of the massless Dirac propagator  can be found in the book "An introduction to quantum field theory"  by Peskin and Schr\"oder  (edition 1995, p. 660) (see \href{https://physics.stackexchange.com/questions/263846/the-analytical-result-for-free-massless-fermion-propagator}{this}) and are given by

\begin{enumerate}

\item  The  representation of the massless Dirac propagator in momentum space is given by

\begin{eqnarray}
\tilde{S}_F(p)=\frac{i \gamma \cdot p}{p^2 +i\varepsilon } .
\end{eqnarray}

\noindent Here the $1/(x+i\varepsilon)$ is a shorthand notation for $1/x\mp i\pi  \delta(x)$.

\item The space-time representation of the massless Dirac propagator is obtained from the momentum representation $D_F(p)= ip^k\gamma_k/(p^2+i\varepsilon)$ as  Fourier transform


\begin{eqnarray}
S_F(x-y)=\int \frac{d^4p}{(2\pi )^4} e^{-ip\cdot (x-y)} \frac {i\gamma \cdot p}{p^2+i\varepsilon } =- \frac{i}{2\pi^2}   \left(\frac{\gamma \cdot (x-y)}{|x-y|^4}\right) .
\end{eqnarray}

\noindent This form conforms  can be deduced from the scaling invariance. 


As one might guess, the propagator has a part, which is a delta function concentrated on the boundary of a double light-cone. The reader can verify this  by calculating the 4-D volume integral of $S_F(m=x-y)$  by using a standard definition of delta function in terms  of 4-D volume integral and the basic rules for integrating delta functions. 

It is advantageous to use  spherical coordinates $(a,\eta, \theta,\phi)$  related to linear Minkowski coordinates $m^k$ by  the equations

$$\begin{array}{llll}
m^0= a\times cosh(\eta),& m^1= a\times sinh(\eta)cos(\theta),& m^2= a\times sinh(\eta)sin(\theta)cos(\phi)&m^3= a\times sinh(\eta)sin(\theta)sin(\phi) .\\
\end{array}
$$   

\noindent $a^2=m^km_k$ defines  the light-cone proper-time.  The line element of the $M^4$ metric  is that for empty Robertson-Walker cosmology $ds^2= da^2-a^2(dr^2/(1+r^2) -r^2d\Omega^2)$.  The 4-D  volume element is $dV_4= a^3r^2 dadr sin(\theta) d\theta d\phi$.

 In the integrand of the volume integral, the dependence on $a$ disappears. The integral of the  spatial part of $m^k\gamma_k$   over angle coordinates  vanishes. The contribution of $m^0\gamma_0$ gives two delta functions corresponding to $m^0\pm r_M$ non-vanishing at the light-cone boundary.

\end{enumerate}

\subsubsection{Deduction of the $H$ propagator}


The separate conservation of lepton and quark numbers forces  the counterpart of the massless propagator, not only in $H$ but also in $M^4$.

\begin{enumerate}  

\item  The $H$ propagator corresponds to the inverse of the massless Dirac operator of $H$.    It  is essential that  the modes have a well-defined $H$-chirality meaning that lepton and quarks numbers  are separately conserved.

\item One could argue that   since  $M^4$ masses for spinor modes  are given by $p^2= \Lambda_n^2$, where $\Lambda_n^2$ is the eigenvalue of $D(CP_2)^2$,  each mode gives rise to a propagator with mass $\Lambda_n^2$.  Since $\Lambda_n$ corresponds to $CP_2$ mass scale, only the covariantly constant right-handed neutrino would propagate above $CP_2$ mass scale! The other modes would propagate only in the $CP_2$ length scale.  Note however   that in $M^4$  massive propagation along light-like  geodesics reduces to massless propagation.

The second problem is that for this view the poles at $p^2= \Lambda_n^2$ do not emerge and must be put in by hand as normalization factors for off-mass-shell states.  The reason is that the $1/D^2$ factor disappears in the matrix elements of $D/D^2+i\epsilon$   between the massive $H$ spinor modes $\Psi$, which are of form $\Psi=D\Phi$. The only exception is the right-handed neutrino, which is covariantly constant in $CP_2$ degrees of freedom. Only the  right-handed neutrino would progate. 

The delicate difference with respect to the ordinary massive propagation in $M^4$ is that the  ordinary  massive propagator is given by $D_F=1/(p^k\gamma_k+m)= (p^k\gamma_k-m)/(p^2-m^2+i\epsilon)$ by $p^2-m^2=(p^k\gamma_k+m)(p^k\gamma_k-m)$. In TGD,  a propagator massive in $M^4$ sense is massless in the 8-D sense and one has $D_F=1/D= D/(D^2+i\epsilon)$.

\item What does the fixing the H-chirality implied by the  separate conservation of quark and lepton numbers mean? The $H$-spinor is a combination of two  spinors, where the $M^4$ and $CP_2$ chiralities $\epsilon_1$ and $\epsilon_2$ are well defined and their product $\epsilon=\epsilon_1\epsilon_2$ is +1 or -1 depending on whether it is a quark or a lepton. The combinations (1,-1) and (-1,1) and on the other hand (1,1) and (-1,-1) are possible.

But the fixed $M^4$ chirality means that propagation is like that for  a massless fermion  in $M^4$!  The mass spectrum for $D(P_2)$ does not show itself in the $M^4$ part of the propagator and the construction of states and of the propagator are completely separate problems. This picture also corresponds to the picture of QCD where massless quarks are assumed as an approximation. 

\item What about the propagation in $CP_2$ degrees of freedom?   It follows directly from the defining condition that   the propagator is a delta function in $CP_2$ degrees of freedom. There is no propagation at all. This just what the Euclidian signature of $CP_2$ suggests.  The matrix elements of the propagator between quantum states constructed from the  conformal representations having $H$ spinor modes as ground states are extremely simple and reduce to the same form as in QFTs  for fermions with internal quantum  numbers.

\end{enumerate}

\subsubsection{Some consequences}

This result has some  highly non-trivial consequences.

\begin{enumerate}

\item  The $H$ propagation includes   the  constraint that the end points of the propagator belong to the space-time surface through the fact that $h_1$   and $h_2$   are on the same spacetime surface. Since there is no   propagation at all in $CP_2$ the  points $s_1$ and $s_2$  are identical!

The condition $s_1=s_2$ at  the spacetime surface gives two complex conditions for $s_2$ when $s_1$ is fixed and the result is a discrete set of points on the spacetime surface! Only in these cases does propagation occur. The theory discretizes itself in $CP_2$ degrees of freedom! Discretization would not be an approximation but a prediction of the theory. This is the basic idea behind the identification of cognitive representation as a  number theoretic discretization using  points of the embedding space for which $H$ coordinates make sense both as real numbers and numbers in an extension of p-adic numbers.

\item The light-like parton orbits  are in a special role now since  the partonic 2-surfaces at the ends  of the parton orbit can have a large number of common points if their $CP_2$  projections are identical. This would explain why they are physically in exceptional position.

The picture of propagation along light-like geodesics  suggest by physical intuition is consistent with the condition $s_1=s_2$ but a light-like geodesic is not needed for the H-propagator. 

\item From this it should be easy to understand what happens in quark and color confinement (gluons are bound states of quarks and antiquarks in TGD). The classical time evolution involves the failure of classical determinism identifiable as p-adic non-determinism \cite{btart}{Langlands2025,HHtwistor}. There is a sequence of analogs of multifurcations involving non-determinism.   In this sequence the number of common points possessed by the initial and final partonic orbits decreases and eventually the propagator amplitude becomes very small.  This means color confinement.

\end{enumerate}




\input{/Users/matpitka/tgdmaterials/texfiles/dirprop1D}

\appendix

\section{About the solutions of Dirac equation in $H$}

This section serves as an Appendix and re-represents information appearing in \cite{allb}{mless}. First the TGD view   of electroweak and color  interactions is compared with the standard view. After that the  general solutions of the Dirac equation in $H$ are discussed.


\subsection{How does the TGD view of standard model interactions differ from the standard model view?}

TGD based vision standard model interactions differs in several respects  from the standard view.

\begin{enumerate}

\item In TGD, elementary particles correspond to closed  monopole flux tubes as analogs of hadronic strings connecting two Minkowskian space-time sheets by Euclidean wormhole contacts. The light-like orbits of wormhole throats (partonic orbits) carry   fermions and antifermions at light curves located at light-like 3-surfaces, which define  interfaces between  Minkowskian string world regions and Euclidean regions identified as deformed  $CP_2$ type extremals.  

\item The basic difference at the level of $H$ spinor fields is that color quantum numbers are  not  spin-like but  are replaced with color partial waves in $CP_2$. Color degrees of freedom are analogous to the rotational degrees of freedom of a rigid body. An infinite number of color partial waves  emerges for both quarks and leptons.  In TGD, color and electroweak degrees of freedom are  strongly correlated as is also clear from the fact that color symmetries correspond to the non-broken symmetries as isometries of $CP_2$ and electroweak symmetries correspond to the holonomies of $CP_2$, which are automatically broken gauge symmetries.

The spectrum of color partial waves  in $H$ is different for U and D type quarks and for charged leptons and neutrinos.  The triality of the partial wave is zero for leptons and 1 {\it resp.} -1 for quarks {\it resp.} antiquarks. At the level of fundamental fermions, which do not correspond as such to fermions as elementary particles, there is a strong violation of isospin symmetry.


\end{enumerate}


The  physical states are constructed using p-adic thermodynamics  \cite{allb}{mless,elvafu} \cite{btart}{padmass2022} for the scaling generator $L_0$ of the conformal symmetries extended to the space-time level and  involve the action of Kac-Moody type algebras. 
The  basic challenge of the state construction of the physical states  is to obtain physical states with correct color quantum numbers.  


\begin{enumerate}

\item  General irrep of $SU(3)$ is labelled by a pair $(p,q)$ of integers, where $p$ {\it  resp.} $q$ corresponds intuitively to the number of quarks {\it  resp.} antiquarks. The dimension of the representation  is $d(p,q)= (1/2)(p+1)(q+1)(p+q+2)$.

The spinors assignable to left and right handed neutrino correspond to representations of color group of type $(p,p)$, where the integers  and only right-handed neutrino allows singlet $(0,0)$ as covariantly constant $CP_2$ spinor mode. $(1,1)$ corresponds to octet 8.  Charged leptons allow representations of type $(3+p,p)$:   $p=0$ corresponds to decuplet 10. 
Note that $(0,3)$ corresponds to $\overline{10}$.

Quarks correspond to irreps of type obtained from leptons by adding one quarks that is replacing  $(p+3,p)$ with $(p+4,p)$ ($p=0$ gives $d=20$) or $(p,p)$ with $(p+1,p)$ ($p=1$ gives $d=42$).    Antiquarks are obtained by replacing $(p,p+3)$ replaced with $(p,p+4)$ and
$(p,p)$ with $(p,p+1)$.


\item Physical  leptons (quarks) are color singlets (triplets).  One can imagine two ways to achieve this.

{\bf Option I}: The  conformal generators act  on the ground state defined by  the spinor harmonic of $H$.  Could the tensor product of the conformal generators with  spinor modes give a color singlet state for leptons and triplet state for quarks?  The constraint that Kac-Moody type generators annihilate the physical states, realizing conformal invariance, might pose severe difficulties. 

In fact, TGD leads to the proposal that there is a hierarchical symmetry breaking for conformal half-algebras containing  a hierarchy of isomorphic sub-algebras with conformal weights coming  as multiplets of the weights of the entire algebra.  This would  make the gauge symmetry of the  subalgebra with weights below given maximal weight to a physical symmetry.

{\bf Option II}: The proposal is that  the   wormhole throats also contain pairs of left- and right-handed neutrinos   guaranteeing that the total electroweak quantum numbers of the string-like closed monopole flux tube representing hadron vanishes. This would make the weak interactions short-ranged with  the range determined by the length of the string-like object.

One must study the tensor products of  $\nu_L\overline{\nu}_R$ and $\overline{\nu_L}\nu_R$ states with the leptonic (quark) spinor harmonic to see whether it is possible to obtain  singlet (triplet) states. The tensor product  of a neutrino octet with a neutrino type spinor contains a color singlet. The tensor product $8\otimes 8 = 1+8_A+8_S+10+\overline{10}+27$ contains $\overline{10}$  and its tensor product with 10 for quark contains a color triplet.

\end{enumerate}



The ground states for  the Super Virasoro representations correspond to spinor harmonics in $M^4 \times CP_2$ characterized by momentum and  color quantum numbers. The correlation between color and  electro-weak quantum numbers is   wrong for the spinor harmonics and these states would be also  hyper-massive. One can imagine two mechanisms.

\begin{enumerate} 

\item The super-symplectic generators could allow to build  color triplet states having negative vacuum  conformal weights, and their values are  such that p-adic massivation is consistent with the predictions of the earlier model differing from the recent one in the quark sector. In the following the construction and the properties of the  color partial waves for fermions and bosons are considered. The discussion follows closely to the discussion of   \cite{bmat}{Pope} .

\item In the concrete model for elementary particles a pair of left- and right-handed neutrino associated with the monopole flux tube screens weak isospin above weak boson Compton length. This pair could be in such color partial waves that  the outcome is leptonic color singlet  and quark color triplet.

\end{enumerate}

In the sequel only the color partial waves in $H$ are discussed. This reduces to the discussion of the eigenstates of $CP_2$ d'Alembertian whose eigestates the spinor harmonics are.






\subsection{General construction of solutions of Dirac operator of $H$}

The construction of the  solutions of massless spinor and other d'Alembertians in  $M^4_+\times CP_2$ is based on the following observations.




\begin{enumerate}

\item  d'Alembertian corresponds to a massless wave equation $M^4\times CP_2$ and thus  Kaluza-Klein picture applies, that is  $M^4_+$ mass is generated from the  momentum in  $CP_2$ degrees of freedom. This implies mass quantization:

\begin{eqnarray}
M^2 =M_n^2\per ,
\end{eqnarray}


\noindent  where $M_n^2$ are eigenvalues of $CP_2$ Laplacian. Here of course, ordinary field theory is considered. In TGD the vacuum  weight changes mass squared spectrum.


\item In order to get a respectable  spinor structure in $CP_2$ one must couple $CP_2$ spinors to an odd integer multiple of the K\"ahler gauge potential. Leptons and quarks correspond to $n=3$ and $n=1$ couplings respectively.  The spectrum of the electromagnetic charge comes out correctly for leptons and quarks.

\item Right handed neutrino is covariantly constant solution of $CP_2$ Laplacian for $n=3$ coupling to  K\"ahler gauge potential whereas right handed "electron" corresponds to the covariantly constant solution for $n=-3$. From the covariant constancy it follows that all solutions of the spinor Laplacian are obtained from these two basic solutions by multiplying with an appropriate solution of the scalar Laplacian coupled to K\"ahler gauge potential with such a  coupling that  a correct total K\"ahler charge results.  Left handed solutions of spinor Laplacian are obtained simply by multiplying right handed solutions with $CP_2$ Dirac operator: in this operation the eigenvalues of the mass squared operator are obviously preserved.


\item The remaining task is to solve  scalar Laplacian coupled to an arbitrary integer multiple of K\"ahler gauge potential. This can be achieved by noticing that the solutions of the massive $CP_2$ Laplacian can be regarded as solutions of $S^5$ scalar Laplacian. $S^5$ can indeed be regarded as a circle bundle over $CP_2$ and massive solutions of $CP_2$ Laplacian correspond to the solutions  of $S^5$ Laplacian with $exp(is\tau)$ dependence on $S^1$ coordinate such that $s$ corresponds to the coupling to the K\"ahler gauge potential:

\begin{eqnarray}
 s=n/2\per .
 \end{eqnarray}
 
\noindent Thus one obtains

\begin{eqnarray} 
D^2_5 &=& (D_{\mu}- iA_{\mu}\partial_{\tau})(D^{\mu}- iA^{\mu}\partial_{\tau}) +\partial_{\tau}^2 
\end{eqnarray}

\noindent so that the eigen values of $CP_2$ scalar Laplacian are

\begin{eqnarray} m^2(s)&=& m_5^2 +s^2 
\end{eqnarray} 

\noindent for the assumed dependence on $\tau$.

\item What remains to do,  is to find the spectrum of $S^5$ Laplacian and this is an easy task.  All solutions of $S^5$ Laplacian can be written as homogenous polynomial functions of $C^3$ complex coordinates $Z^k$ and their complex conjugates and have a decomposition into the representations of $SU(3)$  acting in natural manner in $C^3$.



\item The solutions of the scalar Laplacian belong to the representations $(p,p+s)$ for $s\geq 0$ and to the representations $(p+ \vert s\vert, p)$ of $SU(3)$ for $s\leq 0$.  The eigenvalues  $m^2 (s)$ and degeneracies $d$ are

\begin{eqnarray} m^2(s)&=& \frac{2\Lambda}{3}[p^2+ (\vert s\vert +2) p + \vert s\vert] \per , \per  p>0\per , \nonumber\\ d&=& \frac{1}{2}(p+1)(p+ \vert s\vert +1)(2p+ \vert s\vert +2)\per . \end{eqnarray}


\noindent $\Lambda$ denotes the \blockquote{cosmological constant} of $CP_2$ ($R_{ij}= \Lambda s_{ij}$). 

\end{enumerate}


\subsubsection{Solutions of the leptonic spinor Laplacian}

Right handed solutions of the leptonic   spinor Laplacian are obtained from the  asatz of form 

\begin{eqnarray}
 \nu_R= \Phi_{s=0} \nu^0_R \per , 
 \end{eqnarray}
 
\noindent where $u_R$ is covariantly constant right handed neutrino and $\Phi$ scalar with vanishing K\"ahler charge. Right handed "electron" is obtained from the ansatz 
 
\begin{eqnarray}
 e_R= \Phi_{s=3} e_R^0\per ,
 \end{eqnarray}
 

\noindent where $e_R^0$ is covariantly constant for $n=-3$ coupling to K\"ahler gauge potential so that scalar function must have K\"ahler coupling $s=n/2=3$ a in order to get a  correct K\"ahler charge.  The d'Alembert equation reduces to

\begin{eqnarray} (D_{\mu}D^{\mu} - (1 - \epsilon) \Lambda)\Phi =-m^2\Phi\per ,\nonumber\\ \epsilon (\nu)=1 \per , \per \per \epsilon (e)=-1\per . \end{eqnarray}

\noindent The two additional terms correspond to the curvature scalar term and $J_{kl}\Sigma^{kl}$ terms in spinor   Laplacian. The latter term is proportional to K\"ahler coupling and of different sign for $\nu$ and $e$, which explains the presence of the sign factor $\epsilon$ in the formula.




Right handed neutrinos correspond to $(p,p)$ states with $p\geq 0$ with mass spectrum

\begin{eqnarray} m^2(\nu)&=& \frac{m_1^2}{3}\left[p^2+ 2 p \right] \per ,\per  p\geq 0\per ,\nonumber\\ m_1^2&\equiv& 2\Lambda \per . \end{eqnarray}

\noindent Right handed \blockquote{electrons} correspond to $(p,p+3)$ states with mass spectrum


\begin{eqnarray} m^2(e)&=& \frac{m_1^2}{3}\left[p^2+ 5 p +6\right] \per ,\per  p\geq 0\per . \end{eqnarray}

\noindent Left handed solutions are obtained by operating with $CP_2$ Dirac operator on right handed solutions with one exception: the action of the Dirac operator on the covariantly constant  right handed neutrino ($(p=0,p=0)$ state)  annihilates it.






\subsubsection{Quark spectrum}

Quarks correspond to the second conserved  $H$-chirality of $H$-spinors. The construction of the color partial waves for quarks proceeds along similar lines as for leptons. The K\"ahler coupling corresponds to $n=1$ (and $s=1/2$) and right handed $U$ type quark corresponds to a right handed neutrino. $U$ quark type solutions are constructed as solutions of form

\begin{eqnarray}
U_R=  u_R\Phi_{s==1}\per ,
\end{eqnarray}



\noindent where  $u_R$ possesses the quantum numbers of covariantly constant right handed neutrino with K\"ahler charge $n=3$ ($s=3/2$). Hence   $\Phi_s$ has  $s=-1$.  For $D_R$ one has

\begin{eqnarray}
D_R= d_r\Phi_{s=2}\per .
\end{eqnarray}



\noindent $d_R$ has $s=-3/2$ so that one must have $s= 2$. For $U_R$ the representations $(p+1,p)$ with triality one are obtained and $p=0$ corresponds to color triplet. For $D_R$ the representations $(p,p+2)$ are obtained and color triplet is missing from the spectrum ($p=0$ corresponds to $\bar{6}$).


The $CP_2$ contributions to  masses are given by the formula

\begin{eqnarray} 
m^2(U,p)&=& \frac{m_1^2}{3}\left[p^2+3p+2   \right] \per ,\per  p\geq 0\per , \per \nonumber\\ m^2(D,p)&=& \frac{m_1^2}{3}\left[p^2+ 4 p +4\right] \per ,\per  p\geq 0\per . \end{eqnarray}


\noindent Left handed quarks are obtained by applying Dirac operator to right handed quark states and mass formulas and color partial wave spectrum are the same as for right handed quarks.

The color contributions to p-adic mass squared are integer valued if $m_0^2/3$ is taken as a fundamental p-adic unit of mass squared. This choice has an obvious relevance for p-adic mass calculations since  canonical identification does not commute with a division by integer.  More precisely, the images of number $xp$ in canonical identification has a value of order $1$ when $x$ is a non-trivial rational whereas for $x= np$ the value is $n/p$ and extremely is small for physically interesting primes. This choice does not however affect the spectrum of massless states but can affect the spectrum of light states in case of electro-weak gauge bosons.




\section{Foundational questions}

There are several foundational questions involved. What is the origin of the p-adicity? What is the justification of p-adic thermodynamics and what does it have a counterpart at quantum level? What are the poorly understood aspects of p-adic thermodynamics? What is the understanding of generalized superconformal representations and possible  interpretational and technical problems.


\subsection{Questions related to the p-adic thermodynamics}

The nice feature of both approaches is  that the propagators carry a minimum amount of information  about the quantum states at the 3-surfaces to which the end points of the propagator are associated. Therefore   the construction of physical states with correct total color quantum numbers do not affect the propagators. One also   circumvents the problem how to take into account the massivation by p-adic thermodynamics at the level of the fermionic propagators. 

This construction says nothing about how physical fermions are constructed and how the get their masses via p-adic thermodynamics.


\subsubsection{The origin of p-adic thermodynamics}

The understanding of the origin of p-adic thermodynamics is the first challenge.


\begin{enumerate} 

\item In p-adic thermodynamics,   the scaling generator $L_0$ takes the role of energy as a generator of time translations. Ordinary 2-D conformal invariance requires that physical states are annihilated by $L_0$. All states created by $L_0$ would correspond to gauge degrees of freedom.  This cannot be the case in p-adic thermodynamics.  

In TGD, the conformal algebras are replaced by half-algebras \cite{btart}{wcwsymm}, which allow an  infinite hierarchy of sub-algebras isomorphic to the full algebra, which means that a finite-dimensional  sub-algebra transforms from gauge algebra to a dynamical symmetry algebra and p-adic thermodynamics applies to states in this finite-dimensional state space.

\item  In the 2-D version  of the p-adic thermodynamics, the   ground state is necessarily tachyonic with a negative conformal weight representing the mass squared. The physical  ground state with a vanishing mass squared is obtained by applying conformal generators with a positive conformal weight.  Where do the generators with tachyonic conformal weight come from? 

\begin{enumerate}

\item p-Adic thermodynamics in its  recent formulation \cite{btart}{padmass2022} answers this question by  extending  the conformal algebra  from dimension 2 to  dimension 4 so that one has hypercomplex and complex coordinate and there are  conformal weights associated with both of them. The 4-dimensional conformal invariance  is realized in holography= holomorphy (H-H) vision involving the notion of Hamilton-Jacobi structure in an essential way \cite{brart}{HJ}. There are   conformal weights associated with hypercomplex {\it resp.} complex coordinates and the corresponding conformal weights are non-negative  {\it resp.}  negative. Their sum vanishes for massless ground states.




\item $M^8-H$ duality \cite{btart}{TGDcritics,HHtwistor} provides  further insights to the tachyon problem. In $M^8$, the momentum space  $M^4$ corresponds to a quaternionic subspace of octonions realized as the normal space   of   point  $y$ of  the 4-surfaces $Y^4\subset M^8$. The point $y$  carries the fermion. The 8-D mass squared always vanishes but the $M^4$ part is in general  non-vanishing.   For some points $y\in Y^4$   the normal space $M^4$   is such  that the $E^4$ part of the  8-momentum vanishes and  the particle is massless in $M^4$ sense. This is the counterpart for the cancellation of the two conformal weights at the level of $H$.   

\end{enumerate}

\end{enumerate}


\subsubsection{A possible problem with super-conformal representations}


What  about the higher conformal excitations giving rise to the p-adic thermal mass? There does not seem to be any point or need  to modify the massless propagators since they are associated with the ground state of the super-conformal representations. There is however a problem.

In string models the  super generators $G_n$  of Ramond representation and  $G_{n+1/2}$ of Neveu-Schwarz  representation define the analog of the Dirac operator.  The anticommutators    $\{G_m,G_n^{\dagger}\}$  and $\{G_{m+1/2,G_{n+1/2}^{\dagger}\}$ give Virasoro generators $L_n$.  In the Ramond representation   one has $G_0= p^k\gamma_k +G_{vib,0}$. The Ramond representation   therefore contains the  ordinary Dirac operator acting in cm degrees of freedom. 

In pa-dic thermodynamics \cite{allb}{mless,padmass2022} based on the TGD view of conformal invariance \cite{btart}{wcwsymm}, the Majorana option is not possible.  Since the ordinary gamma matrices do not carry a fermion number, the   Ramond formula of $G$  would involve  generators  with a  vanishing fermion number and non-vanishing fermion number. Same applies also to $G_n$.



\begin{enumerate}

\item A possible solution of the problem is that  super  generators and their Hermitian conjugates  define  two kinds of supergenerators. For Ramond representation, this would also include the counterparts of ordinary gamma matrices.   The algebra creating the states would involve only non-negative conformal weights and   possess a fractal hierarchy of included sub-algebras isomorphic to the entire algebra. 

\item Could $G_0$ is non-hermitian so that both $G_0$ {\it resp.} $G_0^{\dagger}$ carrying a fermion number 1 {\it resp.}  -1 are allowed. The  counterparts of $M^4$ gamma matrices would be analogous to fermionic creation and annihilation operators, which would correspond to a complex structure in center-or mass degrees of freedom. 

Could the Hamilton-Jacobi structure \cite{btart}{HJ} for $M^4$ based on hyper-complex and complex structure  make  this possible.   The  conjugation for complex coordinate  and its hypercomplex counterpart would correspond to the hermitian conjugation for  the operators $\gamma_u$ and $\gamma_w$  producing $\gamma_u^{\dagger}=\gamma_v$ and $\gamma_w^{\dagger}= \gamma_{\overline{w}}$ for both  leptonic and quark-like representations. These gamma matrices would act as quantum operators.

\item Or are only Neveu-Schwarz representations for which all generators $G_{n+1/2}$ can  carry a well-defined fermion number possible?  The problem is that in p-adic mass calculations fermions correspond to Ramond representations with different degeneracies for states with a given conformal weight.  

\item Could zero energy ontology (ZEO) \cite{alb}{zeoquestions} \cite{btart}{zeocriticality} come to   rescue? Could these states  be regarded as  superpositions of pairs of states at opposite boundaries CD with opposite fermion numbers so that there would be no conflict with the superselection rule for fermion numbers. Could the time-like entanglement between  the states at the boundaries of causal diamond (CD)  relate to the  p-adic thermodynamics or rather, its square root. What about the relationship to the model of superconductivity in terms of quantum coherent states of Cooper pairs?

\end{enumerate}



\subsubsection{Questions related to the origin of p-adicity}

One can also make ontological questions related to p-adicity.

\begin{enumerate}

\item What is the origin of the p-adicity? H-H vision \cite{btart}{Langlands2025,HHtwistor} has led to a considerable progress in the attempts to understand the origin of p-adicity and the  p-adic length scale hypothesis. The notion of functional  p-adic numbers suggests how the p-adicization and  adelization  emerge.  One can even ask whether it is  necessary to introduce the p-adic and adelic  variants of the embedding space in this approach.


\item  p-Adic thermodynamics  is a statistical description and must be an approximation to a genuine quantal description.  Quantum TGD can be formally regarded as a square  root of thermodynamics at a quantum critical temperature defined by the  K\"ahler coupling strength  \cite{allb}{UandM,vNeumannnew}. Could it make sense to speak of a square root of p-adic thermodynamics in which thermal probabilities correspond to quantal probabilities. Thermal state would be replaced by a  quantum state.  Very naively, thermodynamic weights would be replaced with their complex square roots, that is ordinary square roots  involving a phase factor.

\end{enumerate}


\subsection{Does 8-dimensionality imply 8-dimensional relativity?}

One cannot avoid the question whether 8-dimensionality implies an 8-D generalization of relativity.


\begin{enumerate}

\item    The completely new element is that at $M^8$ level $M^4$  as normal space is not fixed but dynamical: could one speak  of relativity for mass squared? The Lorentz boosts of this relativity could correspond to the Lorentz group $SO(1,7)$  of $M^8$. Also  $G_2$ as a subgroup affects the $M^4$  mass squared values since it does not leave quaternionic subspace  $M^4$ invariant although it does not affect the energy.

\item Masslessness in the 8-D sense at $M^8$ level forces us to challenge the assumption that the particles always have a well-defined  4-D mass squared. As a matter of fact, p-adic thermodynamics and its possible quantum variant as a square root of thermodynamics \cite{allb}{UandM,vNeumannnew}  already does this. Could one think that mass is more like a transversal or longitudinal momentum squared and  depends on the state of motion characterized by the normal space of $Y^4$?

\item What about color quantum numbers: could 8-D relativity apply even to them.  This would change dramatically the views of color. In $M^8$,  $SO(4)$ acting on $E^4\subset M^8$ is the counterpart of $SU(3)$ acting in $CP_2\subset H$ and also as a  subgroup $SU(3)\subset G_2$ of octonion automorphisms leaving invariant the octonion unit  defining $M^4$ time direction in the normal space.  $M^4$ as a quaternionic normal space  of $Y^4$ varies and therefore also the subgroup $SU(3)\subset G_2$.  Suppose that one replaces normal space-with a new one by deforming $Y^4$ such that the point $y\in E^4$ is not affected. What happens?

I have asked whether  $SO(4)-SU(3)$ duality  between hadron physics and QCD type physics  could correspond to $M^8-H$ duality \cite{btart}{M8H1,M8H2,TGDcritics,HHtwistor}.  Could $M^8-H$ duality map  the 2 Cartan quantum numbers of $SO(4)$ representations  appearing in  $M^8$ spinor harmonics  to those of $SU(3)$ representations of $H$ spinor harmonics. Could the representations of $SU(3)$ be restricted to the $U(2)$ subgroup of  the covering group of $SO(4)$ in accordance with the left-right asymmetry of the standard model? 

Suppose that $SO(1,7)$ and also $G_2$ transformations affect the $SO(4)$  representations. If so, then by $M^8-H$ duality in the proposed  form they also  affect the $SU(3)$ color partial waves for $H$ spinors. By color confinement this is however not easy to observe.

\end{enumerate}



\subsection{Is the proposed picture consistent with $M^8-H$ duality}


$M^8-H$ duality \cite{btart}{TGDcritics,HHtwistor} plays a central role in TGD. Therefore one can also worry about whether    octotwistors, which for physical states must reduce in some sense to quaternionic twistors, are a well-defined notion. The octonionic spinors must reduce to quaternionic spinors for the physical states in some sense. Massless Dirac equation must be satisfied and the  analogs of the  chirality conditions for $H$, $M^4$ and $CP_2$  must make sense.

In the sequel the   massless  octonionic Dirac equation will  be discussed. 

\begin{enumerate}

\item Octonionic and quaternionic spinors are very much analogous to the 2-spinors appearing in the definition of twistors since $\gamma^0$ in the Dirac equation is replaced with a real octonion unit. The masslessness condition is obtained when Minkowski inner product is defined as the real part for the product of octonions. The  counterparts of sigma matrices are identified as quaternions and octonions and matrix representations for the quaternionic units are not used. This allows the introduction of a  commuting imaginary unit $i$ doubling the number of degrees of freedom and replacing octospinors with their complexified counterparts.

\item The octonionic Dirac equation  looks like  the ordinary Dirac equation but with gamma matrices replaced with octonionic units. The quaternionic Dirac equation involves quaternionic units but it is essential that they are not represented as matrices. This allows the introduction of imaginary uni $i$ commuting with the quaternionic and octonionic units and implies double of the degrees of freedom so that one can have analogs of complex spinors.

The octonionic units are analogs of Pauli sigma matrices and the first problem is caused by the lacking anticommutativity of the real unit with other octonion units. The Dirac equation  however makes sense also in this case. 

\item The   8-D masslessness condition   must correspond  to the condition that the  real part of the square of Dirac operator on spinors vanishes. For momentum eigenstates this gives  the usual  algebraic conditions for masslessness.
 
\item  $H$ spinors have a defined $H$-chirality guaranteeing separate conservation of quark and lepton numbers. $H$-chirality $\epsilon$ is a product $\epsilon=\epsilon_1\epsilon_2$  of $M^4$ and $CP_2$ chiralities. Alls these chiralities should be definable also at the level of $M^8$. Also the  octonionic Dirac equation for $H$ spinors should  be consistent with the chirality condition.

\begin{enumerate}

\item The decomposition of octonion units to quaternion  units $\{1,I_k\vert k=1,2,3\}$ and  co-quaternion units  $I_4\{1,I_k\vert k=1,2,3\}$ suggests the identification of the counterpart of $\Gamma_9$.   The matrix $\Gamma_9$  is  defined as the  product of $H$ gamma matrices  satisfies $\Gamma_9^2=-1$, anticommutes with $H$ gamma matrices.   $H$ chirality  corresponds to the eigenvalue of $i\Gamma_9$ equal to $\epsilon=\pm 1$. The eigen spinors  with chirality $\epsilon$ are of the form $(1+\epsilon i\Gamma_9)\Psi_0$.

The spinors with fixed $H$-chirality are tensor products of spinors of fixed $M^4$ chirality and $CP_2$ chirality and the product of these chiralities defines $H$-chirality.

\item The operator $iI_4$, satisfying  $(iI_4)^2=1$, is a good guess for the counterpart of  $\Gamma_9$  for the octonionic spinors.  The octospinors with a fixed $M^8$ chirality $\epsilon$ should be of the form $(1+\epsilon iI_4)\Psi_0$.   It is easy to check that  for an octospinor  of form $\Psi_{\epsilon}= \Psi_a+ I_4\Psi_b$  having a fixed chirality $\epsilon$,  one obtains 

\begin{eqnarray}
\Psi_b=i\epsilon\Psi_a  
\end{eqnarray}

\noindent  so that the  spinor is determined completely by its quaternionic part. Perhaps this might be regarded as a realization of quaterniocity.
 
\item One can decompose also the quaternionic spinors to two parts corresponding to the decomposition to complex subspace spanned by $\{1,I_1\}$ and co-complex subspace spanned by $\{I_2,I_3\}$. This allows us to define $M^4$ chirality and its $E^4$ counterpart.

\end{enumerate}


\item Octonionic Dirac equation for the momentum eigenstates can be  decomposed to a sum of quaternionic and co-quaternionic parts

\begin{eqnarray}
D\Psi= (p^k_1I_k + I_4p_2^k I_k)\Psi=0 \per .
\end{eqnarray}

\noindent The real part of $D^2$ gives  the Minkowskian mass shell condition $p_1^2-p_2^2=0$.  


The Dirac equation for the   $\Psi_{\epsilon}= (1+i\epsilon I_4)\Psi_0$ gives

\begin{eqnarray}
\begin{array}{ll}
D\Psi_0 =(p^k_1I_k + I_4p_2^k I_k)\Psi_0 + \tilde{D}\tilde{\Psi}_0=0 \per .\\
\tilde{D} =\tilde{p}^k_1I_k + I_4\tilde{p}_2^k I_k) \per .
\end{array}
\end{eqnarray}

\noindent Tilde means a conjugation of quaternionic imaginary units. This  gives two separate equations? Are they consistent?  By multiplying the  equation with tildes by $1= -I_2^2$ from left and transporting the second $I_4$  through the equation to right, one obtains the equation $-I_2(p^k_1I_k + I_4p_2^k I_k)\Psi_0I_5=0$. The two equations are therefore consistent.
 
\end{enumerate}


\vm

{\bf Acknowledgements:} The discussions with the Zoom group (Marko Manninen, Tuomas Sorakivi, Ville-Veli Saari and Rode Majakka) have been very useful since they have forced to explain various TGD related concepts.  The attempt to calculate the fermionic Dirac propagator with the assistance of GPT  together with Tuomas  forced to  formulate this notion precisely and this led to some surprising conclusions discussed in this article.



\input{/Users/matpitka/tgdmaterials/bibgrafiat/referencetitle}
\input{/Users/matpitka/tgdmaterials/bibartgrafiat/bibdirprop}
\input{/Users/matpitka/tgdmaterials/bibgrafiat/endgroup}


\end{document}